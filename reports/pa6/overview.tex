\documentclass{hw}
\title{Programming Assignment 6:\\ Optimization}

\usepackage{fancyvrb}
\usepackage{mathpartir}
\usepackage{pervasives}
\usepackage{tikz}
\usetikzlibrary{positioning}

\newcommand{\ir}{intermediate representation}

\begin{document}
\maketitle

\section{Metadata}\label{sec:metadata}
% The fully qualified class name of your main program and any other
% instructions needed to run the program.
We implemented programming assignment $1$ and $2$ in Java, but we implemented
programming assignment $3,4,5$, and $6$ in OCaml. Our main OCaml executable is
implemented in \texttt{main.ml} inside the \texttt{src} directory. To run our
code, you will need to install a couple of packages and OCaml libraries by
running the following:

\begin{center}
\begin{BVerbatim}
sudo apt-get install aspcud m4 unzip
opam install core async oUnit
\end{BVerbatim}
\end{center}

To build our executable, simply run \texttt{make src}. This will use
\texttt{jflex}, \texttt{cup}, and \texttt{javac} and build all of our Java
code. It will use \texttt{corebuild} to build our OCaml code. Invoking our
code manually is hard; instead, we we recommend you use the \texttt{xic} script
which invokes our code with everything configured properly. 
In summary, perform the following:

\begin{center}
\begin{BVerbatim}
sudo apt-get install aspcud m4 unzip
opam install core async oUnit
make src
./xic [flags] <xi_file.xi>...
\end{BVerbatim}
\end{center}

See the \texttt{opam} file for more information regarding our OCaml dependencies.


\section{Summary}\label{sec:summary}
In this programming assignment, we implemented several optimizations for our compiler - register allocation, conditional constant propagation, and partial-redundancy elimination (which also implements common subexpression elimination).
The most challenging part of this assignment was implementing register allocation. Major design decisions for this project include strictly following Appel's implementation of register allocation. %TODO
There are no known bugs or missing functionality in our implementation, modulo a very
small set of corner cases.

\section{Specification}\label{sec:specification}
For this project, we have not deviated from the project specification, given we were allowed
great freedom to choose instructions for IR nodes. The set of x86 instructions that our
compiler may generate is documented in \texttt{asm.mli}.

\section{Design and Implementation}\label{sec:design}
\subsection{Architecture}
Our assembly code generation is implemented by the following modules.
\begin{itemize}
  \item \texttt{tiling}
    This module implements both naive and non-naive tiling for instruction selection.

  \item \texttt{asm}
    This module contains a representation of the x86 assembly language in OCaml. It also
    includes functions to construct all possible x86 instructions that we may generate,
    as well as printers for these instructions.

  \item \texttt{func\_context}
    This module produces function information derived from the body and type of each function.
    This information is used to both assist in instruction selection and to calculate a 
    static stack frame layout for each function.
\end{itemize}

\subsection{Code Design}
We describe a number of major design choices in our assembly generation.    

\subsubsection{Stack frame design}
We compute static stack frames for each function - that is, during the
duration of the function's lifetime, the stack pointer should not
move except when calling other functions. This is computed in the
following manner for a function $f$, supposing that $G$ is the set of functions
that $f$ calls:
  \begin{enumerate}
    \item If the generated abstract assembly body of $f$ contains $n$ fake temps, and
    \item $maxret = max(0, (\max_{g\in G} \text{num\_return\_values}(g)) - 2)$, and
    \item $maxargs = max(0, x)$, where
        \begin{align*}
          x &= (\max_{g\in G} \text{num\_arguments}(g)) + ((maxret > 0) \text{ ? } 1 : 0) - 6)
        \end{align*}
  \end{enumerate}
then the stack frame of f should have $1 + 1 + 9 + n + maxret + maxargs$ words,
with 1 word for the saved rip, 1 word for the saved rbp, and 9 words for all
caller save registers. Note that we include the ternary expression in $maxargs$
to include an argument for the pointer to the space allocated on the stack for
functions called that return more than $2$ arguments. For all $n>2$ return possible
values of functions called, 1 pointer suffices given that we pass in an address to
the base of the allocated stack space. The third return value is placed in this base
address, and all subsequent returns are placed in the subsequent words in the space.
There is also optionally an additional word to keep the stack aligned to 16 bytes.
This word is placed if the static stack frame allocated does not have a number of bytes
that are not aligned to 16 bytes.

Each stack frame is laid out in the following manner:

\begin{center}
  \begin{tabular}{|c|}
    \hline
    Saved instruction pointer \\
    \hline
    Saved base pointer \\
    \hline
    Caller save registers \\
    \hline
    Stack-allocated assembly temps \\
    \hline
    Space for spilled return values \\
    \hline
    Space for spilled function arguments \\
    \hline
  \end{tabular}
\end{center}

Once the function is entered, the stack pointer stays pointed at the base of the
space for spilled function arguments and does not change (except for calls).
This allows access to the space for return values (moving returns from a callee)
and function arguments (moving args to a callee) without knowledge of how many caller
saves/fake temps there are by using rsp as a base.

\subsubsection{Renaming logic for special return and argument registers}
Consider \texttt{Move(dest, src)} with IR Temps regarding \texttt{ARGi} and \texttt{RETi}.

If \texttt{dest = RETi}, then we are the callee returning values to the caller,
so we stick these in the rdi/rsi/pointers passed as args. The callee
knows how many return values it has so it knows how many of the passed
args are pointers. If this function returns $n > 2$ values, then the
first $(n-2)$ arguments are pointers, (i.e. first registers then stack
locations), so look here for where to stick the values.

If \texttt{src = RETi}, then we are the caller moving values from the callee
return. These should be in the allocated multi-return space, which can
be calculated given the height of the block for passing in arguments
to functions via the stack in the stackframe is fixed.

If \texttt{src = ARGi}, then we are the callee moving passed arguments into
the appropriate variable temps. If we return n values for n > 2, then
the first $(n-2)$ arguments to the function are pointers. Thus "\texttt{ARG0}"
in this case is actually the $(n-2+1)$th argument, "\texttt{ARG1}" the $n$th, etc. 
In this case we also need to move the ret pointers into appropriate
fresh temps. This should probably be done in the prologue.

dest = \texttt{ARGi} is not generated in our IR. 

\subsubsection{Register allocation}
We implemented a very naive register allocation. Each abstract assembly register
is given a unique location in the stack. Then, before any abstract register is used,
it is first shuttled from the stack, operated on, and then written back to the stack.
This naive allocation is very inefficient but was easy to implement. For more information,
see src/ocaml/tiling.mli.

\subsection{Programming}
We followed a bottom-up implementation strategy, implementing each pass of our
assmebly generation algorithm independently before finally gluing together each
pass after they had all been tested. This bottom-up approach made it easy to
divide work between team members and maximized modularity.

We first implemented the most naive tiles by simply tiling individual IR nodes.
We then implemented a richer set of tiles, building on top of our initial naive
work. This helped us ensure that our instruction selection is exhaustive, i.e.
that we would be able to produce assembly instructions for all possible IR trees.
It also partitioned our work in such a way that a lot of the most complex reasoning
regarding instruction selection, like handling function prologues and function calls,
were the focus of our early efforts, given we implemented these with naive tiles. 

The most challenging part of the assignment was handling the complexity of the x86
architecture. None of our group members had much experience with this architecture,
so we had to spend a significant amount of time familiarizing ourselves with
its design and quirks. Debugging was especially challenging given that we were
dealing with executing actual binary programs, though this was alleviated in part
by judicious unit testing and use of \texttt{gdb}.

\section{Testing}\label{sec:testing}
As usual, we have implemented a very comprehensive set of tests that range in
scale and scope. First, we have implemented the typical unit tests for each major
code component (e.g. partial-redundancy elimination). This ensured that our functions behaved as
expected and validated many of our assumptions regarding our optimization algorithms.

We also wrote and ran many functional tests. This was done within
an automated framework that required us only to write high-level Xi programs. This
framework performed testing in a novel manner: for each stage of our compiler from
typechecking, AST constant folding, IR generation, IR lowering, IR block reordering,
all the way to assembly generation, both with naive and non-naive tiling, we generate
output and use the Xi interpreter written by Seung Hee for pa4,
the provided IR interpreter, and gcc to link assembly programs and execute them on our
machines. We then diff the outputs from all of these executions. This allowed us to validate that the stages of our compiler were still working from the previous programming assignments.

The testing method that helped us debug the most intricate bugs was generating the assembly and analyzing it to understand what would cause our tests to fail, infinitely loop, etc. We should have used this method earlier on in the project because register allocation is a very involved algorithm and even our unit tests were too high level to isolate where the bugs were occurring.

We pass all of our tests.
\section{Work Plan}\label{sec:workplan}
\begin{itemize}
  \item \textbf{Ralph.}
    Ralph worked on the front-end, set up the data analysis framework, and helped implement and debug register allocation.

  \item \textbf{Alice.}
    Alice worked on register allocation and helped test the generated CFGs and conditional constant propagation.

  \item \textbf{Seung Hee.}
    Seung Hee implemented conditional constant propagation and partial-redundancy elimination.

  \item \textbf{Michael.}
    Michael worked on partial-redundancy elimination, extensive debugging in register allocation, and making the dot files.
\end{itemize}

The work was divided in this way because it parallelized our work in the most
effective manner.

\section{Known Problems}\label{sec:problems}
Currently we do not support UTF-8 as OCaml does not have good support for
unicode escapes. We also sometimes report incorrect file positions in our error
messages.

\section{Comments}\label{sec:comments}
We spent about 120 hours total on the assignment. As usual, we spent over an
order of magnitude more time writing tests and gluing things together than we
did writing actual code. The assignment was hard because the algorithms had many intricate details, testing became much
more challenging, and also it was unclear to see whether all of our generated
code is correct given the complexity of the architecture.

\end{document}
