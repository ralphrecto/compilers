\documentclass[landscape]{hw}
\title{oXi Type System Specification}
\author{\ }
\date{\ }

\usepackage{mathpartir}
\usepackage{pervasives}

% keywords
\newcommand{\meta}[1]{\textsf{#1}}
\newcommand{\key}[1]{\texttt{#1}}

\newcommand{\tint}{\key{int}}
\newcommand{\tbool}{\key{bool}}
\newcommand{\tclass}{\key{C}}
\newcommand{\tempty}{\key{empty}}
\newcommand{\tnull}{\key{null}}
\newcommand{\tarr}[1]{#1\key{[\,]}}
\newcommand{\tunit}{\key{unit}}
\newcommand{\tvar}[1]{\key{var}\ #1}
\newcommand{\tret}[1]{\key{ret}\ #1}
\newcommand{\tfn}[1]{\key{fn}\ #1}

\newcommand{\xlength}[1]{\meta{length(}#1\meta{)}}
\newcommand{\xarr}[1]{\meta{\{}#1\meta{\}}}
\newcommand{\xindex}[2]{#1\meta{[}#2\meta{]}}
\newcommand{\xnull}{\meta{null}}
\newcommand{\xnew}[1]{\meta{new}\ #1}

% subtypes
\newcommand{\asubt}{\leq_{a}}
\newcommand{\subt}{\leq}

% typing
\newcommand{\typ}[2]{#1\colon#2}
\newcommand{\conts}{\Gamma; \Delta_m; \Delta_i; \gamma}
\newcommand{\typed}[2]{\conts \vdash \typ{#1}{#2}}

\begin{document}
\maketitle
\small

\section{Types}
\[
  \begin{array}{rrl}
    \tau & ::= & \tint \\
         & |   & \tbool \\
         & |   & \tclass \\
         & |   & \tarr{\tau} \\
         & |   & \tempty \\
         & |   & \tnull \\
         & |   & \tarr{\tau}
  \end{array}
  \qquad
  \begin{array}{rrl}
    T & ::= & \tau \\
      & |   & \tunit \\
      & |   & (\tau_1, \ldots, \tau_n)
  \end{array}
  \qquad
  \begin{array}{rrl}
    R & ::= & 0 \\
      & |   & 1 \\
  \end{array}
  \qquad
  \begin{array}{rrl}
    \sigma & ::= & \tvar{\tau} \\
           & |   & \tret{T} \\
           & |   & \tfn{T \to T'} \\
  \end{array}
\]

\begin{itemize}
  \item $\Gamma   : \meta{Var} \to \sigma$
  \item $\Delta_x : \meta{Class} \to \meta{Class}_\tau$
  \item $\gamma   : \meta{Class}$
\end{itemize}

\section{Subtyping}
\subsection{Array Subtyping}
\begin{mathpar}
  \inferrule*
  { }
  {\tau \asubt \tau}

  \inferrule*
  { }
  {\tnull \asubt \tempty}

  \inferrule*
  { }
  {\tnull \asubt \tarr{\tau}}

  \inferrule*
  { }
  {\tnull \asubt \tclass}

  \inferrule*
  { }
  {\tempty \asubt \tarr{\tau}}

  \inferrule*
  {\tau_1 \asubt \tau_2 }
  {\tarr{\tau_1} \asubt \tarr{\tau_2}}
\end{mathpar}

\subsection{General Subtyping}
\begin{mathpar}
  \inferrule*
  {\tau_1 \asubt \tau_2}
  {\tau_1 \subt \tau_2}

  \inferrule*
  {\key{A extends B}}
  {\key{A} \subt \key{B}}

  \inferrule*
  {\key{A} \subt \key{B} \\ \key{B} \subt \key{C}}
  {\key{A} \subt \key{C}}
\end{mathpar}

\section{Expressions}
\begin{mathpar}
  \inferrule*
  { }
  {\typed{n}{\tint}}

  \inferrule*
  { }
  {\typed{\meta{true}}{\tbool}}

  \inferrule*
  { }
  {\typed{\meta{false}}{\tbool}}

  \inferrule*
  { }
  {\typed{\meta{string}}{\tarr{\tint}}}

  \inferrule*
  { }
  {\typed{\meta{char}}{\tint}}

  \inferrule*
  {\Gamma(x) = \tvar{\tau}}
  {\typed{x}{\tau}}

  \inferrule*
  {
    \typed{e_1}{\tint} \\
    \typed{e_2}{\tint} \\
    \oplus \in \set{\key{+}, \key{-}, \key{*}, \key{*>>}, \key{/}, \key{\%}}
  }
  {\typed{e_1 \oplus e_2}{\tint}}

  \inferrule*
  {\typed{e}{\tint}}
  {\typed{\key{-}e}{\tint}}

  \inferrule*
  {
    \typed{e_1}{\tint} \\
    \typed{e_2}{\tint} \\
    \oplus \in \set{\key{==}, \key{!=}, \key{<}, \key{<=}, \key{>}, \key{>=}}
  }
  {\typed{e_1 \oplus e_2}{\tbool}}

  \inferrule*
  {\typed{e}{\tbool}}
  {\typed{\key{!}e}{\tbool}}

  \inferrule*
  {
    \typed{e_1}{\tbool} \\
    \typed{e_2}{\tbool} \\
    \oplus \in \set{\key{==}, \key{!=}, \key{\&}, \key{|}}
  }
  {\typed{e_1 \oplus e_2}{\tbool}}

  \inferrule*
  {
    \typed{e_1}{\tau} \\
    \tempty \asubt \tau
  }
  {\typed{\xlength{e}}{\tint}}

  \inferrule*
  {
    \typed{e_1}{\tau_1} \\
    \typed{e_2}{\tau_2} \\
    \tau_1 \asubt \tau_2 \lor \tau_2 \asubt \tau_1 \\
    \oplus \in \set{\key{==}, \key{!=}}
  }
  {\typed{e_1 \oplus e_2}{\tbool}}

  \inferrule*
  { }
  {\typed{\xarr{}}{\tempty}}

  \inferrule*
  {
    \typed{e_i}{\tau_i} \\
    \tau = \lub(\tau_1, \ldots, \tau_n)
  }
  {\typed{\xarr{e_1, \ldots, e_n}}{\tau}}

  \inferrule*
  {
    \typed{e_1}{\tarr{\tau}} \\
    \typed{e_2}{\tint} \\
  }
  {\typed{\xindex{e_1}{e_2}}{\tau}}

  \inferrule*
  {
    \typed{e_1}{\tau_1} \\
    \typed{e_1}{\tau_2} \\
    \tempty \asubt \tau_i \\
    \tau = \lub(\tau_1, \tau_2)
  }
  {\typed{\xindex{e_1}{e_2}}{\tau}}

  \inferrule*
  {
    \typed{f}{\tfn{\tunit \to T}} \\
    T \neq \tunit
  }
  {\typed{f()}{T}}

  \inferrule*
  {
    \typed{f}{\tfn{(\tau_1, \ldots, \tau_n) \to T}} \\
    \typed{e_i}{\sigma_i} \\
    \sigma_i \subt \tau_i \\
    T \neq \tunit
  }
  {\typed{f(e_1, \ldots, e_n)}{T}}

  \inferrule*
  { }
  {\typed{\xnull}{\tnull}}

  \inferrule*
  {\tclass \in \dom(\Delta_m)}
  {\typed{\xnew{\meta{C}}}{\tclass}}

  \inferrule*
  {
    \typed{e}{\tclass} \\
    \Delta_m(\tclass)(f) = \tau
  }
  {\typed{e.f}{\tau}}

  \inferrule*
  {
    \typed{e}{\tclass} \\
    \Delta(\tclass)(f) = \tfn{\tunit \to T} \\
    T \neq \tunit
  }
  {\typed{e.f()}{T}}

  \inferrule*
  {
    \typed{e}{\tclass} \\
    \Delta(\tclass)(f) = \tfn{(\tau_1, \ldots, \tau_n) \to T} \\
    \typed{e_i}{\sigma_i} \\
    \sigma_i \subt \tau_i \\
    T \neq \tunit
  }
  {\typed{f(e_1, \ldots, e_n)}{T}}
\end{mathpar}

\section{Statements}
\end{document}
