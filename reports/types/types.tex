\documentclass[landscape]{hw}
\title{oXi Type System Specification}
\author{\ }
\date{\ }

\usepackage{mathpartir}
\usepackage{pervasives}

\allowdisplaybreaks

% keywords
\newcommand{\meta}[1]{\textsf{#1}}
\newcommand{\key}[1]{\texttt{#1}}

\newcommand{\tint}{\key{int}}
\newcommand{\tbool}{\key{bool}}
\newcommand{\tclass}{\key{C}}
\newcommand{\tempty}{\key{empty}}
\newcommand{\tnull}{\key{null}}
\newcommand{\tarr}[1]{#1\key{[\,]}}
\newcommand{\tunit}{\key{unit}}
\newcommand{\tvar}[1]{\key{var}\ #1}
\newcommand{\tret}[1]{\key{ret}\ #1}
\newcommand{\tfn}[1]{\key{fn}\ #1}
\newcommand{\tbrack}[1]{\key{[}#1\key{]}}

\newcommand{\xlength}[1]{\meta{length(}#1\meta{)}}
\newcommand{\xarr}[1]{\meta{\{}#1\meta{\}}}
\newcommand{\xindex}[2]{#1\meta{[}#2\meta{]}}
\newcommand{\xnull}{\meta{null}}
\newcommand{\xnew}[1]{\meta{new}\ #1}
\newcommand{\xiif}[2]{\meta{if}(#1)\ #2}
\newcommand{\xiifelse}[3]{\meta{if}(#1)\ #2 \ \meta{else}\ #3}
\newcommand{\xiwhile}[2]{\meta{while}(#1)\ #2}
\newcommand{\xibreak}{\meta{break}}
\newcommand{\xireturn}{\meta{return}}

% subtypes
\newcommand{\asubt}{\leq_{a}}
\newcommand{\subt}{\leq}

% typing
\newcommand{\typ}[2]{#1\colon#2}
\newcommand{\conts}[1][]{\Gamma{#1}; \Delta_m; \Delta_i; \gamma; \lambda; \rho}
\newcommand{\typed}[3][]{\conts[#1] \vdash \typ{#2}{#3}}
\newcommand{\lval}[1]{#1\ \meta{lvalue}}
\newcommand{\ok}{\meta{ok}}

\begin{document}
\maketitle

\section{Types}
\[
  \begin{array}{rrl}
    \tau & ::= & \tint \\
         & |   & \tbool \\
         & |   & \tclass \\
         & |   & \tempty \\
         & |   & \tnull \\
         & |   & \tarr{\tau}
  \end{array}
  \qquad
  \begin{array}{rrl}
    T & ::= & \tau \\
      & |   & \tunit \\
      & |   & (\tau_1, \ldots, \tau_n)
  \end{array}
  \qquad
  \begin{array}{rrl}
    R & ::= & 0 \\
      & |   & 1 \\
  \end{array}
  \qquad
  \begin{array}{rrl}
    \sigma & ::= & \tvar{\tau} \\
           & |   & \tfn{T \to T'} \\
  \end{array}
\]

\begin{itemize}
  \item $\Gamma   : \meta{Var} \to \sigma$
  \item $\Delta_x : \meta{Class} \to \meta{Class}_\tau$
  \item $\gamma   : \meta{Class}$
  \item $\lambda  : \set{\top, \bot}$
\end{itemize}

\section{Subtyping}
\subsection{Array Subtyping}
\begin{mathparpagebreakable}
  \inferrule*
  { }
  {\tau \asubt \tau}

  \inferrule*
  { }
  {\tnull \asubt \tempty}

  \inferrule*
  { }
  {\tnull \asubt \tarr{\tau}}

  \inferrule*
  { }
  {\tnull \asubt \tclass}

  \inferrule*
  { }
  {\tempty \asubt \tarr{\tau}}

  \inferrule*
  {\tau_1 \asubt \tau_2 }
  {\tarr{\tau_1} \asubt \tarr{\tau_2}}
\end{mathparpagebreakable}

\subsection{General Subtyping}
\begin{mathparpagebreakable}
  \inferrule*
  {\tau_1 \asubt \tau_2}
  {\tau_1 \subt \tau_2}

  \inferrule*
  {\key{A extends B}}
  {\key{A} \subt \key{B}}

  \inferrule*
  {\key{A} \subt \key{B} \\ \key{B} \subt \key{C}}
  {\key{A} \subt \key{C}}
\end{mathparpagebreakable}

\section{Expressions}
\begin{mathparpagebreakable}
  \inferrule*
  { }
  {\typed{n}{\tint}}

  \inferrule*
  { }
  {\typed{\meta{true}}{\tbool}}

  \inferrule*
  { }
  {\typed{\meta{false}}{\tbool}}

  \inferrule*
  { }
  {\typed{\meta{string}}{\tarr{\tint}}}

  \inferrule*
  { }
  {\typed{\meta{char}}{\tint}}

  \inferrule*
  {\Gamma(x) = \tvar{\tau}}
  {\typed{x}{\tau}}

  \inferrule*
  {
    \typed{e_1}{\tint} \\
    \typed{e_2}{\tint} \\
    \oplus \in \set{\key{+}, \key{-}, \key{*}, \key{*>>}, \key{/}, \key{\%}}
  }
  {\typed{e_1 \oplus e_2}{\tint}}

  \inferrule*
  {\typed{e}{\tint}}
  {\typed{\key{-}e}{\tint}}

  \inferrule*
  {
    \typed{e_1}{\tint} \\
    \typed{e_2}{\tint} \\
    \oplus \in \set{\key{==}, \key{!=}, \key{<}, \key{<=}, \key{>}, \key{>=}}
  }
  {\typed{e_1 \oplus e_2}{\tbool}}

  \inferrule*
  {\typed{e}{\tbool}}
  {\typed{\key{!}e}{\tbool}}

  \inferrule*
  {
    \typed{e_1}{\tbool} \\
    \typed{e_2}{\tbool} \\
    \oplus \in \set{\key{==}, \key{!=}, \key{\&}, \key{|}}
  }
  {\typed{e_1 \oplus e_2}{\tbool}}

  \inferrule*
  {
    \typed{e_1}{\tau} \\
    \tempty \asubt \tau
  }
  {\typed{\xlength{e}}{\tint}}

  \inferrule*
  {
    \typed{e_1}{\tau_1} \\
    \typed{e_2}{\tau_2} \\
    \tau_1 \asubt \tau_2 \lor \tau_2 \asubt \tau_1 \\
    \oplus \in \set{\key{==}, \key{!=}}
  }
  {\typed{e_1 \oplus e_2}{\tbool}}

  \inferrule*
  { }
  {\typed{\xarr{}}{\tempty}}

  \inferrule*
  {
    \typed{e_i}{\tau_i} \\
    \tau = \lub(\tau_1, \ldots, \tau_n)
  }
  {\typed{\xarr{e_1, \ldots, e_n}}{\tau}}

  \inferrule*
  {
    \typed{e_1}{\tarr{\tau}} \\
    \typed{e_2}{\tint} \\
  }
  {\typed{\xindex{e_1}{e_2}}{\tau}}

  \inferrule*
  {
    \typed{e_1}{\tau_1} \\
    \typed{e_1}{\tau_2} \\
    \tempty \asubt \tau_i \\
    \tau = \lub(\tau_1, \tau_2)
  }
  {\typed{\xindex{e_1}{e_2}}{\tau}}

  \inferrule*
  {
    \typed{f}{\tfn{\tunit \to T}} \\
    T \neq \tunit
  }
  {\typed{f()}{T}}

  \inferrule*
  {
    \typed{f}{\tfn{(\tau_1, \ldots, \tau_n) \to T}} \\
    \typed{e_i}{\sigma_i} \\
    \sigma_i \subt \tau_i \\
    T \neq \tunit
  }
  {\typed{f(e_1, \ldots, e_n)}{T}}

  \inferrule*
  { }
  {\typed{\xnull}{\tnull}}

  \inferrule*
  {\tclass \in \dom(\Delta_m)}
  {\typed{\xnew{\meta{C}}}{\tclass}}

  \inferrule*
  {
    \typed{e}{\tclass} \\
    \Delta_m(\tclass)(f) = \tau
  }
  {\typed{e.f}{\tau}}

  \inferrule*
  {
    \typed{e}{\tclass} \\
    \Delta(\tclass)(f) = \tfn{\tunit \to T} \\
    T \neq \tunit
  }
  {\typed{e.f()}{T}}

  \inferrule*
  {
    \typed{e}{\tclass} \\
    \Delta(\tclass)(f) = \tfn{(\tau_1, \ldots, \tau_n) \to T} \\
    \typed{e_i}{\sigma_i} \\
    \sigma_i \subt \tau_i \\
    T \neq \tunit
  }
  {\typed{f(e_1, \ldots, e_n)}{T}}

  \inferrule*
  {
    \typed{e_1}{\key{A}} \\
    \typed{e_2}{\key{B}} \\
    \oplus \in \set{\key{==}, \key{!=}} \\
    ???
  }
  {\typed{e_1 \oplus e_2}{\tbool}}
\end{mathparpagebreakable}

\section{Statements}
\subsection{lvalues}
\begin{mathparpagebreakable}
  \inferrule*
  {
    \Gamma(x) = \tvar{\tau} \\
    x \neq \meta{this}
  }
  {\typed{x}{\lval{\tau}}}

  \inferrule*
  {
    \typed{e_1}{\tarr{\tau}} \\
    \typed{e_2}{\tint}
  }
  {\typed{\xindex{e_1}{e_2}}{\lval{\tau}}}

  \inferrule*
  {
    \typed{e}{\tclass} \\
    \Delta_m(\tclass)(f) = \tau
  }
  {\typed{e.f}{\lval{\tau}}}
\end{mathparpagebreakable}

\subsection{Types}
\begin{mathparpagebreakable}
  \inferrule*
  { }
  {\typed{\tint}{\ok}}

  \inferrule*
  { }
  {\typed{\tbool}{\ok}}

  \inferrule*
  { }
  {\typed{\tempty}{\ok}}

  \inferrule*
  { }
  {\typed{\tnull}{\ok}}

  \inferrule*
  {\tclass \in \dom(\Delta)}
  {\typed{\tclass}{\ok}}

  \inferrule*
  {\typed{\tau}{\ok}}
  {\typed{\tarr{\tau}}{\ok}}
\end{mathparpagebreakable}


\subsection{General Statements}
\begin{mathparpagebreakable}
  \inferrule*
  {
    \typed[]{s_1}{1,\Gamma_1} \\
    \typed[_1]{s_2}{1,\Gamma_2} \\
    \ldots \\
    \typed[_{n-1}]{s_n}{R,\Gamma_n}
  }
  {\typed{\xarr{s_1; \ldots; s_n}}{R, \Gamma}}

  \inferrule*
  {
    \typed{e}{\tbool} \\
    \typed{s}{R, \Gamma'}
  }
  {\typed{\xiif{e}{s}}{1, \Gamma}}

  \inferrule*
  {
    \typed{e}{\tbool} \\
    \typed{s_1}{R_1, \Gamma_1} \\
    \typed{s_2}{R_2, \Gamma_2}
  }
  {\typed{\xiifelse{e}{s_1}{s_2}}{\lub(R_1, R_2), \Gamma}}

  \inferrule*
  {
    \typed{e}{\tbool} \\
    \Gamma; \Delta_m; \Delta_i; \gamma; \top; \rho \vdash \typ{s}{R, \Gamma'}
  }
  {\typed{\xiwhile{e}{s}}{1, \Gamma}}

  \inferrule*
  { }
  {\Gamma; \Delta_m; \Delta_i; \gamma; \top; \rho \vdash \typ{\xibreak}{0, \Gamma}}

  \inferrule*
  {\Gamma(f) = \tfn{\tunit \to \tunit}}
  {\typed{f()}{1,\Gamma}}

  \inferrule*
  {
    \Gamma(f) = \tfn{(\tau_1, \ldots, \tau_n) \to \tunit} \\
    \typed{e_i}{\sigma_i} \\
    \sigma_i \subt \tau_i
  }
  {\typed{f(e_1, \ldots, e_n)}{1,\Gamma}}

  \inferrule*
  {
    \typed{e}{\tclass} \\
    \Delta(\tclass)(f) = \tfn{\tunit \to \tunit}
  }
  {\typed{e.f()}{1,\Gamma}}

  \inferrule*
  {
    \typed{e}{\tclass} \\
    \Delta(\tclass)(f) = \tfn{(\tau_1, \ldots, \tau_n) \to \tunit} \\
    \typed{e_i}{\sigma_i} \\
    \sigma_i \subt \tau_i
  }
  {\typed{e.f(e_1, \ldots, e_n)}{1,\Gamma}}

  \inferrule*
  { }
  {\Gamma;\Delta_m;\Delta_i;\gamma;\lambda;\tunit \vdash \typ{\xireturn}{0, \Gamma}}

  \inferrule*
  {
    \typed{e_i}{\sigma_i} \\
    \sigma_i \subt \tau_i
  }
  {
    \Gamma;\Delta_m;\Delta_i;\gamma;\lambda;(\tau_1, \ldots, \tau_n)
    \vdash
    \typ{\xireturn\ (e_1, \ldots, e_n)}{0, \Gamma}
  }

  \inferrule*
  {
    \typed{e_1}{\lval{\tau_1}} \\
    \typed{e_2}{\tau_2} \\
    \tau_2 \subt \tau_1
  }
  {
    \typed{e_1 = e_2}{1,\Gamma}
  }

  \inferrule*
  {
    \typed{\tau_i}{\ok} \\
    x_i \notin \dom(\Gamma) \\
    x_i \neq x_j \\
    x_i \neq \meta{this} \\
  }
  {
    \typed{\typ{x_1}{\tau_1}, \ldots, \typ{x_n}{\tau_n}}{1,\Gamma[x_i \mapsto \tau_i]}
  }

  \inferrule*
  {
    \typed{\tau}{\ok} \\
    x \notin \dom(\Gamma) \\
    x \neq \meta{this} \\
    \typed{e_i}{\tint}
  }
  {
    \typed{
      \typ{x}{\tau\tbrack{e_1}\ldots\tbrack{e_n}\tbrack{\,}\ldots\tbrack{\,}}
    }{
      1,\Gamma[x \mapsto \tvar{\tau\tbrack{\,}\ldots\tbrack{\,}}]
    }
  }

  \inferrule*
  {
    \typed{\typeof{d_i}}{\ok} \\
    \typed{e}{(\tau_1, \ldots, \tau_n)} \\
    \tau_i \subt \typeof(d_i) \\
    \dom(\Gamma) \cap \varsof(d_i) = \emptyset \\
    \set{\meta{this}} \cap \varsof(d_i) = \emptyset \\
    \varsof{d_i} \cap \varsof(d_j) = \emptyset
  }
  {\typed{d_1, \ldots, d_n = e}{1,\Gamma[x \mapsto \tvar{\typeof(d_i)}]}}
\end{mathparpagebreakable}

\end{document}
