\documentclass{hw}
\title{Programming Assignment 5:\\ Assembly Code Generation}

\usepackage{fancyvrb}
\usepackage{mathpartir}
\usepackage{pervasives}
\usepackage{tikz}
\usetikzlibrary{positioning}

\newcommand{\ir}{intermediate representation}

\begin{document}
\maketitle

\section{Metadata}\label{sec:metadata}
% The fully qualified class name of your main program and any other
% instructions needed to run the program.
We implemented programming assignment $1$ and $2$ in Java, but we implemented
programming assignment $3,4$, and $5$ in OCaml. Our main OCaml executable is
implemented in \texttt{main.ml} inside the \texttt{src} directory. To run our
code, you will need to install a couple of packages and OCaml libraries by
running the following:

\begin{center}
\begin{BVerbatim}
sudo apt-get install aspcud m4 unzip
opam install core async oUnit
\end{BVerbatim}
\end{center}

To build our executable, simply run \texttt{make src}. This will use
\texttt{jflex}, \texttt{cup}, and \texttt{javac} and build all of our Java
code. It will use \texttt{corebuild} to build our OCaml code. Invoking our
code manually is hard; instead, we we recommend you use the \texttt{xic} script
which invokes our code with everything configured properly.

In summary, perform the following:

\begin{center}
\begin{BVerbatim}
sudo apt-get install aspcud m4 unzip
opam install core async oUnit
make src
./xic [flags] <xi_file.xi>...
\end{BVerbatim}
\end{center}

\section{Summary}\label{sec:summary}
In this programming assignment, we generated x86-64 assembly for the Xi programming language.
The most challenging part of this assignment was handling the complexity of the target
architecture, especially debugging our generated assembly. Major design decisions for this
project include naive register allocation (i.e. all temps to stack); calculating a static
stack frame layout for each function; and implementing both a naive set and a richer set of
tiles. There are no known bugs or missing functionality in our implementation.

\section{Specification}\label{sec:specification}
For this project, we have not deviated from the project specification, given we were allowed
great freedom to choose instructions for IR nodes. The set of x86 instructions that our
compiler may generate is documented in \texttt{asm.mli}.

\section{Design and Implementation}\label{sec:design}
\subsection{Architecture}
Our assembly code generation is implemented by the following modules.
\begin{itemize}
  \item \texttt{tiling}
    This module implements both naive and non-naive tiling for instruction selection.

  \item \texttt{asm}
    This module contains a representation of the x86 assembly language in OCaml. It also
    includes functions to construct all possible x86 instructions that we may generate,
    as well as printers for these instructions.

  \item \texttt{func\_context}
    This module produces function information derived from the body and type of each function.
    This information is used to both assist in instruction selection and to calculate a 
    static stack frame layout for each function.
\end{itemize}

\subsection{Code Design}
We describe a number of major design choices in our assembly generation.    

\subsubsection{Stack frame design}
We compute static stack frames for each function - that is, during the
duration of the function's lifetime, the stack pointer should not
move except when calling other functions. This is computed in the
following manner for a function $f$, supposing that $G$ is the set of functions
that $f$ calls:
  \begin{enumerate}
    \item If the generated abstract assembly body of $f$ contains $n$ fake temps, and
    \item $maxret = max(0, (\max_{g\in G} \text{num\_return\_values}(g)) - 2)$, and
    \item $maxargs = max(0, x)$, where
        \begin{align*}
          x &= (\max_{g\in G} \text{num\_arguments}(g)) + ((maxret > 0) \text{ ? } 1 : 0) - 6)
        \end{align*}
  \end{enumerate}
then the stack frame of f should have $1 + 1 + 9 + n + maxret + maxargs$ words,
with 1 word for the saved rip, 1 word for the saved rbp, and 9 words for all
caller save registers. Note that we include the ternary expression in $maxargs$
to include an argument for the pointer to the space allocated on the stack for
functions called that return more than $2$ arguments. For all $n>2$ return possible
values of functions called, 1 pointer suffices given that we pass in an address to
the base of the allocated stack space. The third return value is placed in this base
address, and all subsequent returns are placed in the subsequent words in the space.
There is also optionally an additional word to keep the stack aligned to 16 bytes.
This word is placed if the static stack frame allocated does not have a number of bytes
that are not aligned to 16 bytes.

Each stack frame is laid out in the following manner:

\begin{center}
  \begin{tabular}{|c|}
    \hline
    Saved instruction pointer \\
    \hline
    Saved base pointer \\
    \hline
    Caller save registers \\
    \hline
    Stack-allocated assembly temps \\
    \hline
    Space for spilled return values \\
    \hline
    Space for spilled function arguments \\
    \hline
  \end{tabular}
\end{center}

Once the function is entered, the stack pointer stays pointed at the base of the
space for spilled function arguments and does not change (except for calls).
This allows access to the space for return values (moving returns from a callee)
and function arguments (moving args to a callee) without knowledge of how many caller
saves/fake temps there are by using rsp as a base.

\subsubsection{Renaming logic for special return and argument registers}
Consider \texttt{Move(dest, src)} with IR Temps regarding \texttt{ARGi} and \texttt{RETi}.

If \textt{dest = RETi}, then we are the callee returning values to the caller,
so we stick these in the rdi/rsi/pointers passed as args. The callee
knows how many return values it has so it knows how many of the passed
args are pointers. If this function returns $n > 2$ values, then the
first $(n-2)$ arguments are pointers, (i.e. first registers then stack
locations), so look here for where to stick the values.

If \texttt{src = RETi}, then we are the caller moving values from the callee
return. These should be in the allocated multi-return space, which can
be calculated given the height of the block for passing in arguments
to functions via the stack in the stackframe is fixed.

If \texttt{src = ARGi}, then we are the callee moving passed arguments into
the appropriate variable temps. If we return n values for n > 2, then
the first $(n-2)$ arguments to the function are pointers. Thus "\texttt{ARG0}"
in this case is actually the $(n-2+1)$th argument, "\texttt{ARG1}" the $n$th, etc. 
In this case we also need to move the ret pointers into appropriate
fresh temps. This should probably be done in the prologue.

dest = \texttt{ARGi} is not generated in our IR. 

\subsubection{Register allocation}
%% TODO

\subsection{Programming}
We followed a bottom-up implementation strategy, implementing each pass of our
assmebly generation algorithm independently before finally gluing together each
pass after they had all been tested. This bottom-up approach made it easy to
divide work between team members and maximized modularity.

We first implemented the most naive tiles by simply tiling individual IR nodes.
We then implemented a richer set of tiles, building on top of our initial naive
work. This helped us ensure that our instruction selection is exhaustive, i.e.
that we would be able to produce assembly instructions for all possible IR trees.
It also partitioned our work in such a way that a lot of the most complex reasoning
regarding instruction selection, like handling function prologues and function calls,
were the focus of our early efforts, given we implemented these with naive tiles. 

The most challenging part of the assignment was handling the complexity of the x86
architecture. None of our group members had much experience with this architecture,
so we had to spend a significant amount of time familiarizing ourselves with
its design and quirks. Debugging was especially challenging given that we were
dealing with executing actual binary programs, though this was alleviated in part
by judicious unit testing and use of \texttt{gdb}.

\section{Testing}\label{sec:testing}
As usual, we have implemented a very comprehensive set of tests that range in
scale and scope. First, we have implemented the typical unit tests for each major
part of our program (e.g. tiling).



%% TODO

However, this manual approach to unit testing often overfits our algorithms.
For example, consider unit testing block reordering in this way. We generate a
set of input blocks and an expected reordered set of blocks and compare the
two. This works, but the exact order of the output blocks is not specified. If
we implement a different trace select heuristic, then our test cases could all
fail. In order to overcome this problem, we implemented property based testing
for certain algorithms. That is, we encoded the output properties of our
algorithms as simple predicates in OCaml. For example, any output blocks are
valid so long as the first block in the output is the first block in the input,
the blocks form a permutation of the input blocks, and each trace within the
reordered blocks is a valid trace. We then enumerate a large number of input
blocks, generate the output and assert that they satisfy the predicates.
Testing in this way makes our tests future proof and also less tedious to
write.

Most interestingly, we also implemented (and unit tested) an Xi interpreter.
This allows to perform property based testing of our entire code generation
pass. We manually write out a large number of Xi programs of various complexity
and then for each assert that the result of interpreting the program with our
Xi interpreter is the same as interpreting the intermediate code generated from
it using the IR interpreter. This form of testing captures the correctness of
our compiler at a very high level and makes testing quick and flexible.

All our tests pass.

\section{Work Plan}\label{sec:workplan}
\begin{itemize}
  \item \textbf{Ralph.}
    Ralph worked on the front-end, implementing code generation, and worked on
    connecting the Java IR interpreter with our Xi interpreter in order to make
    testing nice and easy.

  \item \textbf{Alice.}
    Alice worked on implementing code generation for both statements and
    expressions and also worked on thoroughly testing the most challenging
    parts of the code generation.

  \item \textbf{Seung Hee.}
    Seung Hee implemented lowering, constant folding, block reordering, and the
    Xi interpreter. She also worked on tested all three and on testing code
    generation.

  \item \textbf{Michael.}
    Michael tested and debugged lowering, constant folding, block reordering,
    and the Xi interpreter.
\end{itemize}
Ralph and Alice focused mainly on code generation while Seung Hee and Michael
focused on lowering, block reordering, constant folding, and Xi evaluation. The
work was divided in this way because none of these tasks depended on each other
and because code generation was considerably more complex than any of the other
parts.

\section{Known Problems}\label{sec:problems}
Currently we do not support UTF-8 as OCaml does not have good support for
unicode escapes. We also sometimes report incorrect file positions in our error
messages.

\section{Comments}\label{sec:comments}
We spent about 120 hours total on the assignment. As usual, we spent over an
order of magnitude more time writing tests and gluing things together than we
did writing actual code. The assignment was hard because testing becomes much
more challenging. There is no easy way to test that generated IR is correct.

\end{document}
