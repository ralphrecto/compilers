\documentclass{hw}
\title{Programming Assignment 1:\\ Implementing Lexical Analysis}
\usepackage{fancyvrb}

\begin{document}
\maketitle

\section{Metadata}\label{sec:metadata}
% The fully qualified class name of your main program and any other
% instructions needed to run the program.
\begin{center}
\begin{BVerbatim}
make src
./xic --lex <xi_file.xi>...
\end{BVerbatim}
\end{center}

Our main executable is implemented in \texttt{mjw297.Main.java} which is found
in the \texttt{src/mjw297} directory. To build our executable, simply run
\texttt{make src}. Our Makefile depends only on \texttt{javac}; all our
dependencies are packaged in the \texttt{lib} directory, and we don't depend on
any external dependencies. Invoking \texttt{mjw297.Main} can be tricky because
you have to include the JARs inside of \texttt{lib} in your classpath. To run
our lexer, we recommend you use the \texttt{xic} script which invokes
\texttt{mjw297.Main} with everything configured properly.

\section{Summary}\label{sec:summary}
In this programming assignment, we implemented a lexer in Java for the Xi
programming language using the JFlex lexer generator and CUP parser generator.
Our major design decisions involve the design of symbol and exception classes,
the choice of language and libraries, and the lexing of a few nefarious Xi
programs. The most challenging aspect of the assignment was the handling of
non-printable characters inside of comments, strings, and characters. For
example, the form feed character `\verb$\f$' is a non-printable character that
is difficult to lex correctly in various contexts. There are no known problems
with our implementation.

\section{Specification}\label{sec:specification}
In this section we make explicit our interpretation of the Xi language
specification and discuss various extensions to it which we have implemented.

\subsection{Line Endings and Whitespace}
The Xi language specification is ambiguous about the definition of a newline
referring to it only informally as a ``newline''. We define a line terminator
to be a carriage return `\verb$\r$', a newline `\verb$\n$', or both
`\verb$\r\n$'. If a `\verb$\r\n$' is present, it is counted as a single
newline. This definition of line terminators is consistent with the Java
Language
Specification\footnote{\url{https://docs.oracle.com/javase/specs/jls/se8/html/jls-3.html\#jls-3.4}}.
Similarly we define whitespace to be any of `\verb$\r$', `\verb$\n$',
`\verb$\r\n$', `\verb$ $', `\verb$\t$', `\verb$\f$'. This also consistent with
the Java Language
Specification\footnote{\url{https://docs.oracle.com/javase/specs/jls/se8/html/jls-3.html\#jls-3.6}}.

\subsection{Comments}
The Xi language specification states that a comment is \texttt{//} followed by
any sequence of characters until a newline character. We interpret a newline
character to be any line terminator as defined above. We have also made a small
revision and allow comments to be terminated by the end of a file in addition
to a line terminator. This rule avoids the rather counter-intuitive scenario
where a file ends in a comment and leads to a lexing error.

\subsection{Characters in Comments, Strings, and Characters}
Xi input files are UTF-8 encoded, though keywords and identifiers are made up
of only ASCII characters. This leads to the question of what characters are
allowed inside of comments, strings, and character literals. We allow any
character that is not a line terminator, including exotic unicode characters,
to appear in a comment, string, or character literal. Allowing exotic
characters simplifies the lexer and is also flexible.

\subsection{Unicode Escape Sequences}
The programming assignment mandates that we support hex escape sequences inside
of string and character literals. For example, the string \verb$"\x64"$ is
lexed into the string \texttt{"d"}. In addition to hex escape sequences, we
have also added unicode escape sequences of the form
\verb$\u[a-fA-F0-9][a-fA-F0-9][a-fA-F0-9][a-fA-F0-9]$. For example, the string
\verb$"\u0064"$ lexes to \texttt{"d"}.

\subsection{Escape Characters}
The Xi language specification says to support a ``reasonable'' set of character
escapes. We have elected to support \verb$\t$, \verb$\b$, \verb$\n$, \verb$\r$,
\verb$\f$, \verb$\'$, \verb$\"$, and \verb$\\$.

\section{Design and Implementation}\label{sec:design}
\subsection{Architecture}
\subsection{Code Design}
\subsection{Programming}

\section{Testing}\label{sec:testing}
\section{Work Plan}\label{sec:workplan}
\section{Known Problems}\label{sec:problems}
\section{Comments}\label{sec:comments}
\end{document}
