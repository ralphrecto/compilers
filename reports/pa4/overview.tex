\documentclass{hw}
\title{Programming Assignment 4:\\ Intermediate Code Generation}

\usepackage{fancyvrb}
\usepackage{mathpartir}
\usepackage{pervasives}
\usepackage{tikz}
\usetikzlibrary{positioning}

\newcommand{\ir}{intermediate representation}

\begin{document}
\maketitle

\section{Metadata}\label{sec:metadata}
% The fully qualified class name of your main program and any other
% instructions needed to run the program.
We implemented programming assignment $1$ and $2$ in Java, but we implemented
programming assignment $3$ and $4$ in OCaml. Our main OCaml executable is
implemented in \texttt{main.ml} inside the \texttt{src} directory. To run our
code, you will need to install a couple of packages and OCaml libraries by
running the following:

\begin{center}
\begin{BVerbatim}
sudo apt-get install aspcud m4 unzip
opam install core async oUnit
\end{BVerbatim}
\end{center}

To build our executable, simply run \texttt{make src}. This will use
\texttt{jflex}, \texttt{cup}, and \texttt{javac} and build all of our Java
code. It will use \texttt{corebuild} to build our OCaml code. Invoking our
code manually is hard; instead, we we recommend you use the \texttt{xic} script
which invokes our code with everything configured properly.

In summary, perform the following:

\begin{center}
\begin{BVerbatim}
sudo apt-get install aspcud m4 unzip
opam install core async oUnit
make src
./xic [flags] <xi_file.xi>...
\end{BVerbatim}
\end{center}

\section{Summary}\label{sec:summary}
In this programming assignment, we implemented intermediate code generation for
the Xi programming language. The most challenging aspect of the assignment was
generating code for complex array operations, including multi-dimensional array
declarations and array concatenation. This assignment didn't involve many
design decisions; the \ir, code generation algorithm, lowering algorithm, and
block reordering algorithm were all taken from the notes. Though, we did
implement novel forms of testing. There are no known problems with our
implementation.

\section{Specification}\label{sec:specification}
For this project, we have not deviated from the project specification at all.
We have also not implemented any extensions. The spec clarifications we made in
previous assignments still stand.

\section{Design and Implementation}\label{sec:design}
\subsection{Architecture}
Our intermediate code generation is implemented by the following modules.
\begin{itemize}
  \item \texttt{ast\_optimization}
    This module implements constant folding at the AST level. It does not
    interact with any \ir.

  \item \texttt{ir}
    This module defines the type of our \ir. Though, this assignment was
    implemented in OCaml, we translated and adopted the types provided in the
    Java release code.

  \item \texttt{ir\_generation}
    This module is responsible for intermediate code generation, lowering,
    constant folding, and basic block reordering. All the algorithms are
    implemented as described in the lecture notes or in the textbook.

  \item \texttt{ir\_printer}
    This module pretty prints our \ir into the appropriate Sexp format.

  \item \texttt{xi\_interpreter}
    This module implements an Xi interpreter. The interpreter takes in a
    type-checked AST and evaluates it to a value. This module is used solely
    for testing, as described below.
\end{itemize}

\subsection{Code Design}
All the algorithms we employ are described in detail in the lecture notes or in
the textbook, so we do not repeat a description of them here. Most of the
algorithms are implemented using vanilla OCaml pattern matching and do not
involve any sort of complex data structure or design pattern.

The one exception is basic block reordering. Block reordering required us to
design a graph data structure and to choose a heuristic for selecting traces.
We opted for a very simple, purely functional graph data structure based on an
adjacency list representation of a graph. Each block in the code is represented
by its label. Each node is paired with a list of its neighbors. Moreover, we
store the graph as simple lists of pairs of lists. This data structure is not
very efficient; many or all of the operations on the graphs are slow. We could
have represented the graphs using sets and maps, but we opted not to. We felt
the simplicity of our graph representation outweighs the benefits of a fast
graph data structure and we had no evidence that block reordering was a
bottleneck of the compiler.

Second, we opted for a very simple trace selection heuristic. Given an ordered
list of blocks, we generate the longest trace rooted at the first block always
taking false branches for conditional jumps. After we generate a trace, we
remove the blocks in the trace from the graph and start the next trace at the
first remaining block. This heuristic is very simple and very easy to
implement, however, it does not produce the longest possible traces. We could
employ a heuristic where we select nodes with the lowest in-degree. We opted
not do implement this heuristic because we felt it would greatly complicate our
code. However, it is possible in the future to implement this optimization.

\subsection{Programming}
We followed a bottom-up implementation strategy, implementing each pass of our
code generation algorithm independently before finally gluing together each
pass after they had all been tested. This bottom-up approach made it easy to
divide work between team members and maximized modularity.

The most challenging part of the assignment was generating code for operations
involving arrays. For example, the declaration of a multidimensional array
involves a recursive function that repeatedly allocates and initializes memory.
Moreover, concatenating two arrays involves reallocating and copying memory in
intricate ways. The lowering, block reordering, and constant folding were much
less challenging.

\section{Testing}\label{sec:testing}
As usual, we have implemented a very comprehensive set of tests that range in
scale and scope. First, we have implemented the typical unit tests for each
pass of our algorithm. In fact, certain algorithms like the block reordering
algorithm were divided into smaller self-contained subroutines that were
thoroughly tested. In these unit tests, we manually construct an input and
expected output and assert that they are equal.

However, this manual approach to unit testing often overfits our algorithms.
For example, consider unit testing block reordering in this way. We generate a
set of input blocks and an expected reordered set of blocks and compare the
two. This works, but the exact order of the output blocks is not specified. If
we implement a different trace select heuristic, then our test cases could all
fail. In order to overcome this problem, we implemented property based testing
for certain algorithms. That is, we encoded the output properties of our
algorithms as simple predicates in OCaml. For example, any output blocks are
valid so long as the first block in the output is the first block in the input,
the blocks form a permutation of the input blocks, and each trace within the
reordered blocks is a valid trace. We then enumerate a large number of input
blocks, generate the output and assert that they satisfy the predicates.
Testing in this way makes our tests future proof and also less tedious to
write.

Most interestingly, we also implemented (and unit tested) an Xi interpreter.
This allows to perform property based testing of our entire code generation
pass. We manually write out a large number of Xi programs of various complexity
and then for each assert that the result of interpreting the program with our
Xi interpreter is the same as interpreting the intermediate code generated from
it using the IR interpreter. This form of testing captures the correctness of
our compiler at a very high level and makes testing quick and flexible.

All our tests pass.

\section{Work Plan}\label{sec:workplan}
\begin{itemize}
  \item \textbf{Ralph.}
    Ralph worked on the front-end, implementing code generation, and worked on
    connecting the Java IR interpreter with our Xi interpreter in order to make
    testing nice and easy.

  \item \textbf{Alice.}
    Alice worked on implementing code generation for both statements and
    expressions and also worked on thoroughly testing the most challenging
    parts of the code generation.

  \item \textbf{Seung Hee.}
    Seung Hee implemented lowering, constant folding, block reordering, and the
    Xi interpreter. She also worked on tested all three and on testing code
    generation.

  \item \textbf{Michael.}
    Michael tested and debugged lowering, constant folding, block reordering,
    and the Xi interpreter.
\end{itemize}
Ralph and Alice focused mainly on code generation while Seung Hee and Michael
focused on lowering, block reordering, constant folding, and Xi evaluation. The
work was divided in this way because none of these tasks depended on each other
and because code generation was considerably more complex than any of the other
parts.

\section{Known Problems}\label{sec:problems}
Currently we do not support UTF-8 as OCaml does not have good support for
unicode escapes. We also sometimes report incorrect file positions in our error
messages.

\section{Comments}\label{sec:comments}
We spent about 120 hours total on the assignment. As usual, we spent over an
order of magnitude more time writing tests and gluing things together than we
did writing actual code. The assignment was hard because testing becomes much
more challenging. There is no easy way to test that generated IR is correct. We
really appreciated the very awesome lecture notes describing each part of code
generation!

\end{document}
